\documentclass[a4paper,10pt]{article}
\usepackage[utf8]{inputenc}
\usepackage{amsmath}
\usepackage{amsfonts}
\usepackage{amssymb}
\usepackage[english]{babel}
\setlength{\parindent}{0cm}
\usepackage{setspace}
\usepackage{mathpazo}
\usepackage{listings}
\usepackage{subfig}
\usepackage{graphicx}
\usepackage{wasysym} 
\usepackage{booktabs}
\usepackage{verbatim}
\usepackage{ulem}
\usepackage{enumerate}
\usepackage{hyperref}
\usepackage{stmaryrd}
\usepackage[a4paper,
left=1.8cm, right=1.8cm,
top=2.0cm, bottom=2.0cm]{geometry}
\usepackage{tabularx}
%\usepackage{tikz}
%\usetikzlibrary{trees,petri,decorations,arrows,automata,shapes,shadows,positioning,plotmarks}


\newcommand{\rf}{\right\rfloor}
\newcommand{\lf}{\left\lfloor}
\newcommand{\tabspace}{15cm}
\newcommand{\N}{\mathbb{N}}
\newcommand{\Z}{\mathbb{Z}}

\begin{document}
\begin{center}
\Large{Cognitive Algorithms: Assignment 4} \\
\end{center}
\begin{tabbing}
Tom Nick \hspace{2cm}\= - 340528\\
Maximilian Bachl \> - 341455 \\
\end{tabbing}

\section*{Exercise 2}

The dimensionality of $X_\textrm{train}$ is $D_\textrm{X} \times D_\textrm{X}$.\\
The dimensionality of $W$ is $D_\textrm{X} \times D_\textrm{Y}$.\\
The dimensionality of $Y_\textrm{test}$ is $D_\textrm{X} \times N_\textrm{te}$.

\section*{Exercise 3}

The training set contains 5000 data points. The test set has 5255 data points. There are 192 e
electrodes used in the experiment.

\section*{Exercise 4}

Of course the training data performs better and the predictions are more accurate. But especially in the diagram at the bottom which shows time and y-axis makes it clear that the prediction is sometimes very inappropriate.

\section*{Exercise 5}

When you use the non-logarithmized data you clearly see that the performance difference between training and test data becomes less.

Actually the smaller difference arises from the fact that the OLS performs worse for its own training data than with logarithmized data.

\section*{Exercise 6}

Yes, it is obvious that stronger impulse does not increase the actual hand movement by the same amount. The true relationship is a logarithmic one like it is common for many human senses and motions.

\end{document}
